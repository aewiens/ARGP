\documentclass[11pt]{article}
\usepackage{amsmath}
\usepackage[margin=0.9cm]{geometry}

\begin{document}

\section{Ordinary Gaussian Process Modeling}
\begin{itemize}
\item A Gaussian process (GP) is a continuous collection of random variables,
any finite number of which have a joint Gaussian distribution.
\item In other words, each GP is a distribution over functions whose
values on X$= \{\mathbf{x}_1,...,\mathbf{x}_n\}$
are normally distributed with mean vection $\mu(\mathrm{X})$ and covariance
matrix $\Sigma(\mathrm{X, X})$.
\item Using $Y$ to denote some Gaussian process, this can be written as
follows,
\begin{align*}
	Y(\cdot)
=
	\mathcal{GP}
	\left(
		\mu(\mathbf{x}),
		\Sigma(\mathbf{x}, \mathbf{x}')
	\right)
&& \implies &&
	Y(\mathrm{X})
=
	\mathcal{N}
	\left(
		\mu(\mathrm{X}),
		\Sigma(\mathrm{X, X})
	\right)
\end{align*}
\item where it is important to make the distinction that $Y(\cdot)$ is a
Gaussian stochastic process characterized by mean function $\mu(\mathbf{x})$
and covariance function $\Sigma(\mathbf{x}, \mathbf{x'})$, whereas Y(X) is a
Gaussian probability distribution in $n$ dimensions, that is characterized by
mean vector $\mu(\mathrm{X})$ and covariance matrix $\Sigma(\mathrm{X, X})$.
\item 
\end{itemize}

\end{document}
